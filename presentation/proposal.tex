\documentclass[10pt, letterpaper]{article}

%%%%%%%%%%%%%%%%%%%%%%%%%%%%%%%%%%%%%%%%%%%%%%%%%%%%%%%%%%%%%%%%%%%%%%
%%%%%%%%%%%%%%%%%%%%%%%%%%%%%%%% FONT %%%%%%%%%%%%%%%%%%%%%%%%%%%%%%%%
%%%%%%%%%%%%%%%%%%%%%%%%%%%%%%%%%%%%%%%%%%%%%%%%%%%%%%%%%%%%%%%%%%%%%%

\usepackage[scaled]{uarial}
\renewcommand*\familydefault{\sfdefault} 
\usepackage[T1]{fontenc}
\usepackage{microtype}

%%%%%%%%%%%%%%%%%%%%%%%%%%%%%%%%%%%%%%%%%%%%%%%%%%%%%%%%%%%%%%%%%%%%%%
%%%%%%%%%%%%%%%%%%%%%%%%%% MARGINS & SPACING %%%%%%%%%%%%%%%%%%%%%%%%%
%%%%%%%%%%%%%%%%%%%%%%%%%%%%%%%%%%%%%%%%%%%%%%%%%%%%%%%%%%%%%%%%%%%%%%

\usepackage[left=1cm, right=1cm, top=1cm, bottom=1cm]{geometry}

\usepackage{setspace}
\setlength{\parindent}{0pt}
\setlength{\parskip}{1em}

\usepackage{enumitem}
\setlist[itemize]{%
	 			  topsep=0pt, 
	              partopsep=0pt, 
	              leftmargin=*,	
	              parsep=0pt,
}

%%%%%%%%%%%%%%%%%%%%%%%%%%%%%%%%%%%%%%%%%%%%%%%%%%%%%%%%%%%%%%%%%%%%%%
%%%%%%%%%%%%%%%%%%%%%%%%%%%%%%% COLORS %%%%%%%%%%%%%%%%%%%%%%%%%%%%%%%
%%%%%%%%%%%%%%%%%%%%%%%%%%%%%%%%%%%%%%%%%%%%%%%%%%%%%%%%%%%%%%%%%%%%%%

\usepackage{xcolor}
\definecolor{darkPinkHighlight}{HTML}{E0249A}

%%%%%%%%%%%%%%%%%%%%%%%%%%%%%%%%%%%%%%%%%%%%%%%%%%%%%%%%%%%%%%%%%%%%%%
%%%%%%%%%%%%%%%%%%%%%%%%%%%%%% HYPER-REF %%%%%%%%%%%%%%%%%%%%%%%%%%%%%
%%%%%%%%%%%%%%%%%%%%%%%%%%%%%%%%%%%%%%%%%%%%%%%%%%%%%%%%%%%%%%%%%%%%%%

\usepackage{color}
\usepackage{hyperref}
\hypersetup{
	pdftoolbar=true,                        
	pdfmenubar=false,                        
	pdffitwindow=false,                      
	pdfstartview={FitH},                    
	pdftitle={Graphical},    
	pdfauthor={Abhinav Chand, 
		       Tristan Freiberg,
		       Astrid Olave,},    
	pdfcreator={Abhinav Chand, 
				Tristan Freiberg,
				Astrid Olave,},
	pdfproducer={Abhinav Chand, 
				 Tristan Freiberg,
				 Astrid Olave,},
	pdfnewwindow=true,                       
	colorlinks=true,
	linkcolor={black},                      
	citecolor={darkPinkHighlight},
	filecolor={black},
	urlcolor={darkPinkHighlight},         
}

%%%%%%%%%%%%%%%%%%%%%%%%%%%%%%%%%%%%%%%%%%%%%%%%%%%%%%%%%%%%%%%%%%%%%%
%%%%%%%%%%%%%%%%%%%%%%%%%% HEADERS & FOOTERS %%%%%%%%%%%%%%%%%%%%%%%%%
%%%%%%%%%%%%%%%%%%%%%%%%%%%%%%%%%%%%%%%%%%%%%%%%%%%%%%%%%%%%%%%%%%%%%%

\pagestyle{empty}

%%%%%%%%%%%%%%%%%%%%%%%%%%%%%%%%%%%%%%%%%%%%%%%%%%%%%%%%%%%%%%%%%%%%%%
%%%%%%%%%%%%%%%%%%%%%%%%%%%%%%% MACROS %%%%%%%%%%%%%%%%%%%%%%%%%%%%%%%
%%%%%%%%%%%%%%%%%%%%%%%%%%%%%%%%%%%%%%%%%%%%%%%%%%%%%%%%%%%%%%%%%%%%%%

\renewcommand{\title}[1]{%
	\begin{center}
		{\Large \bfseries \uppercase{#1}}
	\end{center}
}

\renewcommand{\section}[1]{%
	    \vspace{\parskip}
		{\large \bfseries\uppercase{#1}}
}

\newcommand{\itemTitle}[1]{%
	{\bfseries{#1}}
}

%%%%%%%%%%%%%%%%%%%%%%%%%%%%%%%%%%%%%%%%%%%%%%%%%%%%%%%%%%%%%%%%%%%%%%
%%%%%%%%%%%%%%%%%%%%%%%%%%%%%% CONTENT %%%%%%%%%%%%%%%%%%%%%%%%%%%%%%%
%%%%%%%%%%%%%%%%%%%%%%%%%%%%%%%%%%%%%%%%%%%%%%%%%%%%%%%%%%%%%%%%%%%%%%

\begin{document}
	
\title{Temporal Graphs for Music Recommendation Systems}
	
\section{Dataset}
	
\begin{itemize}
\item 
\href{https://github.com/shenyangHuang/TGB/blob/main/tgb/datasets/dataset_scripts/tgbn-genre.py}{\tt{tgbn-genre}} \cite{tgbn-genre}. Paraphrasing \cite{H_web:2023} (\url{https://tgb.complexdatalab.com/docs/nodeprop/#tgbn-genre}): a bipartite and weighted interaction network between users and music genres, representing users and music genres as nodes, where an interaction denotes a user listening to a music genre at a given time. Edge weights indicate the percentage of a song belonging to a specific genre. The dataset is derived by linking songs from the LastFM-song-listens dataset with music genres from the million-song dataset. 

\end{itemize}
	
\section{Industry value}

\begin{itemize}
\item \itemTitle{Improved recommendation systems:} We aim to enhance recommendation systems for streaming services like Spotify, Netflix, and YouTube. By predicting which set of music/movie/content genres a user will interact with the most, we can personalize recommendations, improve user experience, and increase user engagement and retention.
\end{itemize}
	
\section{Key Stakeholders}
\begin{itemize}
\item Spotify, Netflix, YouTube, and similar streaming services.
\item End users (listeners and viewers).
\end{itemize}
	
\section{Key Performance Indicators}

\begin{itemize}
\item \itemTitle{NDCG@10:} Model performance will be evaluated using mean normalized  discounted cumulative gain of the top 10 items (see \cite{H:2023}).
\item \itemTitle{Leaderboard:} Does the model perform well enough to earn a place on the relevant TGB {\href{https://tgb.complexdatalab.com/docs/leader_nodeprop/\#tgbn-genre}{leaderboard}} (see \cite{H_web:2023}).
\item \itemTitle{User engagement:} [Hypothetical.] User engagement metrics such as time spent on platform, number of interactions, feedback, etc.
\item \itemTitle{Retention rate:} [Hypothetical.] Impact of personalized recommendations on user retention and satisfaction.
\item \itemTitle{Business impact:} [Hypothetical.] Increased subscriptions, ad revenue, customer loyalty, etc.
\end{itemize}
	
\section{Proposal Overview}

\begin{itemize}
\item \itemTitle{Objective:} Develop a machine learning model to 
predict user interactions with music genres over the next week.
\item \itemTitle{Approach:} Use advanced graph-based algorithms 
to model user behavior patterns and preferences.
\item \itemTitle{Expected outcome:} A scalable and well-performing prediction system that enhances existing recommendation algorithms and drives business growth for streaming platforms.
\end{itemize}
	
\section{Conclusion}

Enhancing recommendation systems through predictive analytics can  revolutionize the user experience in the streaming industry. By leveraging the \href{https://tgb.complexdatalab.com/docs/nodeprop/}{\tt{tgbn-genre}} dataset \cite{tgbn-genre}, we aim to create actionable insights and strategic recommendations that empower businesses to deliver personalized content and stay ahead in a competitive market.

%%%%%%%%%%%%%%%%%%%%%%%%%%%%%%%%%%%%%%%%%%%%%%%%%%%%%%%%%%%%%%%%%%%%%%
%%%%%%%%%%%%%%%%%%%%%%%%%%%%% REFERENCES %%%%%%%%%%%%%%%%%%%%%%%%%%%%%
%%%%%%%%%%%%%%%%%%%%%%%%%%%%%%%%%%%%%%%%%%%%%%%%%%%%%%%%%%%%%%%%%%%%%%

\section{References}

\begingroup

\renewcommand{\section}[2]{}%

\begin{thebibliography}{3}
	
\bibitem{H:2023}
Huang, S., et al.
\href{https://doi.org/10.48550/arXiv.2307.01026}
{``Temporal graph benchmark for machine learning on temporal graphs.''} {\em Advances in Neural Information Processing Systems}, 2023. Preprint: \url{arXiv:2307.01026}, 2023.
	
\bibitem{H_GH:2023}
Huang, S., et al. 
\href{https://github.com/shenyangHuang/TGB}
{``TGB.''}
GitHub Repository. \url{https://github.com/shenyangHuang/TGB}, 2023. 
Accessed May 14, 2024.

\bibitem{H_web:2023}
Huang, S., et al.
\href{https://tgb.complexdatalab.com/}
{``Temporal Graph Benchmark.''}
\url{https://tgb.complexdatalab.com/}, 2023.
Accessed May 14, 2024.

\bibitem{tgbn-genre}
Huang, S., et al.
\href{https://github.com/shenyangHuang/TGB/blob/main/tgb/datasets/dataset_scripts/tgbn-genre.py}
{``{\tt{tgbn-genre}} dataset.''}
\url{https://github.com/shenyangHuang/TGB/blob/main/tgb/datasets/dataset_scripts/tgbn-genre.py},
 2023. 
Accessed May 14, 2024.
	
\end{thebibliography}

\endgroup	
	
\end{document}	