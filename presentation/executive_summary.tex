\documentclass[10pt, letterpaper]{article}

%%%%%%%%%%%%%%%%%%%%%%%%%%%%%%%%%%%%%%%%%%%%%%%%%%%%%%%%%%%%%%%%%%%%%%
%%%%%%%%%%%%%%%%%%%%%%%%%%%%%%%% FONT %%%%%%%%%%%%%%%%%%%%%%%%%%%%%%%%
%%%%%%%%%%%%%%%%%%%%%%%%%%%%%%%%%%%%%%%%%%%%%%%%%%%%%%%%%%%%%%%%%%%%%%

\usepackage[scaled]{uarial}
\renewcommand*\familydefault{\sfdefault} 
\usepackage[T1]{fontenc}
\usepackage{microtype}

%%%%%%%%%%%%%%%%%%%%%%%%%%%%%%%%%%%%%%%%%%%%%%%%%%%%%%%%%%%%%%%%%%%%%%
%%%%%%%%%%%%%%%%%%%%%%%%%% MARGINS & SPACING %%%%%%%%%%%%%%%%%%%%%%%%%
%%%%%%%%%%%%%%%%%%%%%%%%%%%%%%%%%%%%%%%%%%%%%%%%%%%%%%%%%%%%%%%%%%%%%%

\usepackage[left=1cm, right=1cm, top=1cm, bottom=1cm]{geometry}

\usepackage{setspace}
\setlength{\parindent}{0pt}
\setlength{\parskip}{1em}

\usepackage{enumitem}
\setlist[itemize]{%
	 			  topsep=0pt, 
	              partopsep=0pt, 
	              leftmargin=*,	
	              parsep=0pt,
}

%%%%%%%%%%%%%%%%%%%%%%%%%%%%%%%%%%%%%%%%%%%%%%%%%%%%%%%%%%%%%%%%%%%%%%
%%%%%%%%%%%%%%%%%%%%%%%%%%%%%%% COLORS %%%%%%%%%%%%%%%%%%%%%%%%%%%%%%%
%%%%%%%%%%%%%%%%%%%%%%%%%%%%%%%%%%%%%%%%%%%%%%%%%%%%%%%%%%%%%%%%%%%%%%

\usepackage{xcolor}
\definecolor{darkPinkHighlight}{HTML}{E0249A}

%%%%%%%%%%%%%%%%%%%%%%%%%%%%%%%%%%%%%%%%%%%%%%%%%%%%%%%%%%%%%%%%%%%%%%
%%%%%%%%%%%%%%%%%%%%%%%%%%%%%% HYPER-REF %%%%%%%%%%%%%%%%%%%%%%%%%%%%%
%%%%%%%%%%%%%%%%%%%%%%%%%%%%%%%%%%%%%%%%%%%%%%%%%%%%%%%%%%%%%%%%%%%%%%

\usepackage{color}
\usepackage{hyperref}
\hypersetup{
	pdftoolbar=true,                        
	pdfmenubar=false,                        
	pdffitwindow=false,                      
	pdfstartview={FitH},                    
	pdftitle={Graphical},    
	pdfauthor={Abhinav Chand, 
		       Tristan Freiberg,
		       Ishika Ghosh,, 
		       Astrid Olave},    
	pdfcreator={Abhinav Chand, 
				Tristan Freiberg,
				Ishika Ghosh,, 
				Astrid Olave},
	pdfproducer={Abhinav Chand, 
				 Tristan Freiberg,
				 Ishika Ghosh, 
				 Astrid Olave},
	pdfnewwindow=true,                       
	colorlinks=true,
	linkcolor={black},                      
	citecolor={darkPinkHighlight},
	filecolor={black},
	urlcolor={darkPinkHighlight},         
}

%%%%%%%%%%%%%%%%%%%%%%%%%%%%%%%%%%%%%%%%%%%%%%%%%%%%%%%%%%%%%%%%%%%%%%
%%%%%%%%%%%%%%%%%%%%%%%%%% HEADERS & FOOTERS %%%%%%%%%%%%%%%%%%%%%%%%%
%%%%%%%%%%%%%%%%%%%%%%%%%%%%%%%%%%%%%%%%%%%%%%%%%%%%%%%%%%%%%%%%%%%%%%

\pagestyle{empty}

%%%%%%%%%%%%%%%%%%%%%%%%%%%%%%%%%%%%%%%%%%%%%%%%%%%%%%%%%%%%%%%%%%%%%%
%%%%%%%%%%%%%%%%%%%%%%%%%%%%%%% MACROS %%%%%%%%%%%%%%%%%%%%%%%%%%%%%%%
%%%%%%%%%%%%%%%%%%%%%%%%%%%%%%%%%%%%%%%%%%%%%%%%%%%%%%%%%%%%%%%%%%%%%%

\renewcommand{\title}[1]{%
	\begin{center}
		{\Large \bfseries \uppercase{#1}}
	\end{center}
}

\renewcommand{\section}[1]{%
	    \vspace{\parskip}
		{\large \bfseries\uppercase{#1}}
}

\newcommand{\itemTitle}[1]{%
	{\bfseries{#1}}
}

%%%%%%%%%%%%%%%%%%%%%%%%%%%%%%%%%%%%%%%%%%%%%%%%%%%%%%%%%%%%%%%%%%%%%%
%%%%%%%%%%%%%%%%%%%%%%%%%%%%%% CONTENT %%%%%%%%%%%%%%%%%%%%%%%%%%%%%%%
%%%%%%%%%%%%%%%%%%%%%%%%%%%%%%%%%%%%%%%%%%%%%%%%%%%%%%%%%%%%%%%%%%%%%%

\begin{document}
	
\title{Musical preference predictions}

\section{Motivation}

``Nearly 616.2 million people listen to their favorite artists or discover new ones via online streaming platforms" \cite{statista}. Hence, streaming platforms seek to increase enhance the user experience by offering personalized music recommendations. At the same time, music streaming services are able to track individual preferences meticulously, thus, a growing volume of data on  multiple user's musical preferences is available.
In response, we aimed to develop a recommendation model that predicts the genres a user is expected to like.

\section{Datasets}

We worked with a temporal evolving graph from Temporal Graph Benchmark (TGB) \cite{H:2023} datasets. TGB provides two data sets: First, a graph with 1000 users and 500 music genres represented as nodes and weighted edges that indicates a user listens to a music genre at a given time. The temporal graph evolves over the span of 4 years. The dataset posed a challenge since the description given by the creators of the dataset did not match it. Additionally, it has anomalies and multiple duplicates. A second dataset derives from the first, a dynamical matrix of users and genres where the interaction of each user and each genre is normalized over the span of a week. We train our models on this second dataset. Nevertheless, we explore the first dataset to comprehend the second one.

\section{Approach}

Our primary goal is to predict the the interaction of each user and each genre on the following week. We begin by ...

\section{Results}

\section{Conclusion}




\begingroup

\renewcommand{\section}[2]{}%

\begin{thebibliography}{3}

\bibitem{statista}
Götting M. 
``Music streaming worldwide - statistics \& facts". 10 Jan 2024, \url{https://www.statista.com/topics/6408/music-streaming/#topicOverview} Accessed May 28, 2024. 
	
\bibitem{H:2023}
Huang, S., et al.
\href{https://doi.org/10.48550/arXiv.2307.01026}
{``Temporal graph benchmark for machine learning on temporal graphs.''} {\em Advances in Neural Information Processing Systems}, 2023. Preprint: \url{arXiv:2307.01026}, 2023.
	
	
\end{thebibliography}

\endgroup	
	
\end{document}	
